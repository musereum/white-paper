\documentclass[12pt]{report}
\ProvidesPackage{musereum}

\usepackage{amsmath}
\usepackage{amsfonts}
\usepackage{amssymb}
\usepackage{booktabs}
\usepackage{pgfplots}

\usepackage{tikz}
\usetikzlibrary{calc,trees,positioning,arrows,chains,fit,shapes.geometric,%
    decorations.pathreplacing,decorations.pathmorphing,shapes,%
    matrix,shapes.symbols}


\usepackage{amsmath}
\usepackage{graphicx}
\usepackage{hyperref}
\hypersetup{
    colorlinks=true,
    linkcolor=blue,
    filecolor=magenta,      
    urlcolor=cyan,
}
%Russian-specific packages
%--------------------------------------
\usepackage[T2A]{fontenc}
\usepackage[utf8]{inputenc}
\usepackage[russian]{babel}
%--------------------------------------
 
%Hyphenation rules
%--------------------------------------
\usepackage{hyphenat}
\hyphenation{ма-те-ма-ти-ка вос-ста-нав-ли-вать}
%--------------------------------------

\usepackage{multicol}
\setlength{\columnsep}{1.2cm}

\usepackage[a4paper, total={7in, 9.1in}]{geometry}

\tikzstyle{event} = [trapezium, trapezium left angle=70, trapezium right angle=110, minimum width=3cm, minimum height=1cm, text centered, draw=black]
\tikzstyle{inout} = [rectangle, rounded corners, minimum width=3cm, minimum height=1cm,text centered, draw=black]
\tikzstyle{rect} = [rectangle, minimum width=3cm, minimum height=1cm, text centered, draw=black]
\tikzstyle{check} = [diamond, minimum width=3cm, minimum height=1cm, text centered, draw=black]
\tikzstyle{arrow} = [thick,->,>=stealth]

\pgfdeclarelayer{background}
\pgfdeclarelayer{foreground}
\pgfsetlayers{background,main,foreground}

%\usepackage{musereum}
\usepackage{import}

\setlength{\parindent}{0em}
\setlength{\parskip}{0.65em}
\renewcommand{\baselinestretch}{1.1}

\definecolor{light-gray}{gray}{0.95}
\def\code#1{\colorbox{light-gray}{\texttt{#1}}}

\title{Musereum}
\author{Artem Aler Dan}
\date{\today}

\begin{document}
\maketitle
\pagebreak
\tableofcontents
\pagebreak

\chapter{Техническое описание платформы}

\section{Обзор}
\label{tech-review}
\begin{multicols}{2}
Задачи протокола «Musereum» необходимо разделить на составляющие:
\begin{itemize}
\item Обеспечение единого состояния системы без необходимости централизации вычислений для обеспечения отказоустойчивости;
\item Децентрализованное хранилище большого объема данных без возможности цензуры путем блокировки единого (централизованного) источника данных;
\item Предоставление доказательства целостности системы для аудирования исторических изменений;
\item Интерфейс внесения изменений в будущее состояние системы.
\end{itemize}
\vfill\null
\columnbreak
Ведение и изменение единого состояния системы «Musereum» без необходимости централизации вычислений гарантируется использованием концепции «блокчейн» с использованием виртуальной машины Ethereum (EVM) \textit{(Подробнее в \hyperref[tech-blockchain]{Блокчейн})}.
Для хранение большого объема данных (аудио-треки, метаданные, текстовое и графическое описание) утилизируется технология «IPFS» \textit{(Подробнее в \hyperref[tech-storage]{Децентрализованное хранилище.})}
\end{multicols}
\pagebreak
\subsection{Соглашение о наименовании}
\label{tech-review-naming}
\begin{multicols}{2}
Здесь и далее по тексту все сущности с определенным содержанием или интерфейсом будут закреплять за собой краткое математическое наименование с общим правилом: \textit{заглавная латинская буква, но не N, B, Z}, например: 

\begin{equation}
Y \ \text{is Notary}
\end{equation}

Все глобально определенные контракты будут закреплять за собой строчный греческий символ, например, контракт управления списком нотариусов – $\nu$ и специальную запись ассоциированную с конкретным функционалом контракта: 

\begin{equation}
\nu \ \text{is Notary Registry Contract}
\end{equation}
\begin{equation}
\nu_{vote} \ \text{is vote contract method}
\end{equation}
%\vfill\null
%\columnbreak

Множества всех известных сущностей определяются как \textit{принятое наименование сущности в black-board записи, но не $\mathbb{N}, \mathbb{B}, \mathbb{Z}$}, например: список нотариусов:
\begin{equation}
\mathbb{Y} \ \text{is set of all known Notaries}
\end{equation}

$\mathbb{N}$ – множество всех целых чисел в диапазоне $(0; 2^{256} - 1)$. $\mathbb{N}_1$ – подмножество $\mathbb{N}$ без 0.

$\mathbb{Z}$ – множество всех целых чисел в диапазоне $(-2^{255}; +2^{255} - 1)$. 

$\mathbb{B}$ – множество всех байтов $(0; 255)$. 

$\mathbb{X}^{i}$ – множество размерности $i$ где каждый элемент множества пренадлежит $\mathbb{X}$, например:
\begin{equation}
\mathbb{B}^{32} \text{ is a set of } \mathbb{B} \text{ with size } 32
\end{equation}

Получение элемента множества может быть записано двумя способами: подстрочным индексом или в квадратных скобках:
\begin{equation}
\begin{aligned}
\mathbb{Y}_2 = \mathbb{Y}[2] \\
\mathbb{Y}_n = \mathbb{Y}[n] 
\end{aligned}
\end{equation}

Внешние данные представлены в формулах как элементы множества $\Gamma$ с удобно читаемыми наименованиями (или сокращениями если это оговорено), например: глобальное время в UNIX формате:
\begin{equation}
\Gamma_{time} \ \text{is world time variable}
\end{equation}

Все временные переменные в формулах записываются как строчная латинская буква, например:
\begin{equation}
\begin{aligned}
\text{Let } a \text{ is a second notary in list of notaries} \\
a = \mathbb{Y}[2]
\end{aligned}
\end{equation}
\end{multicols}

\subsection{Математические функции и символы}
\label{tech-review-math}
%\begin{multicols}{2}
Во всех формулах используется $|\mathbb{Y}|$ как операция взятия количества элементов множества $\mathbb{Y}$.

Для сокращения и/или удобства прочтения математической нотации вводятся следующие соглашения и функции:

\textbf{Поиск индекса в множестве} – $\mathcal{I}$:
\begin{equation}
\begin{aligned}
\text{Let } a \text{ is a set of values: } \\
a = {100, 200, 300} \\
f(v, x) = \{i \ | \ i \in \mathbb{N},  i > |v|, v_i = x \} \\
\mathcal{I}(v, x) = \begin{cases}
	min(f(v,x)), & \text{if } |f(v, x)| > 0 \\
	-1, & \text{overwise}
\end{cases}
\\
\\
r1 = \mathcal{I}(a, 200) = 1 \\
r2 = \mathcal{I}(a, 400) = -1 \\
\end{aligned}
\end{equation}

\textbf{Определения именованного кортежа} – $\mathcal{O}$:
\begin{equation}
\begin{aligned}
\text{Let } a \text{ is a set of names: } \\
a = \{name, age, height\} \\
\text{and let } b \text{ is a set of indexes: } \\
b = \{0, 1, 2\} = \{x \ | \ x \in \mathbb{N}, x < |a|\} \\
\\
\text{finaly } r \text{ is a target named tuple, where} \\
r = \mathcal{O}(a) = \mathcal{O}(name, age, height) \\ 
r = \{x_i \ | \ i \in  a, x_i = 0 \} \\
\end{aligned}
\end{equation}

\textbf{Определения кортежа события} – $\mathcal{E}$:\hfill\null\linebreak
События специальный именнованный кортеж, содержащий в себе кроме целевых данных, общие для любого события: \code{event}, \code{txHash}, \code{txIndex}, \code{blockHash}, \code{blockHeight}, \code{address}.
\begin{equation}
\begin{aligned}
\text{Let } a \text{ is a set of default event fields: } \\
a = \{event, \\
txHash, \\
txIndex, \\
blockHash, \\
blockHeight, \\
address\} \\
\text{Let } b \text{ is a set of extra event fields: } \\
b = \{asset, index\} \\
\\
\text{finaly } e \text{ is a target event tuple, where} \\
r = \mathcal{E}(b) \\
r = \mathcal{O}(\{asset, index\} \ \cup \ a)
\end{aligned}
\end{equation}
%\end{multicols}

\pagebreak
\section{Архитектура платформы}
\label{tech-arch}
\begin{multicols}{2}
Основная задача платформы предоставить пользователям возможность для:
\begin{enumerate}
\item Чтения и интерпретации данных 
\item Внесения изменений
\item Поиска данных
\item Аудита и получения доказательства отсутствия фальсификации данных
\end{enumerate}
Для предоставления всех вышеуказанных возможностей платформа разделена на логические слои: 
\begin{itemize}
	\item \hyperref[tech-arch-underlayer]{Хранение и учет}
	\begin{itemize}
		\item \hyperref[tech-blockchain]{Blockchain}
		\item \hyperref[tech-storage]{Децентрализованное хранилище}
	\end{itemize}
	\item \hyperref[tech-arch-connect]{Слой доступа на чтение и запись}
	\begin{itemize}
		\item \hyperref[tech-arch-connect-nodes]{Публичные узлы}
		\item \hyperref[tech-arch-connect-validators]{Узлы нотариусы}
	\end{itemize}
	\item \hyperref[tech-arch-interfaces]{Интерфейс взаимодействия}
	\begin{itemize}
		\item \hyperref[tech-arch-interfaces-dapp]{Децентрализованные приложения}
		\item \hyperref[tech-arch-interfaces-api]{Application Programming Interface}
	\end{itemize}
\end{itemize}
\end{multicols}

\def\Interface{Интерфейс взаимодействия}
\def\Connect{Слой доступа}
\def\Underlayer{Слой хранения и учета}

\def\DApps{Децентрализованные приложения}
\def\Api{API}
\def\Requests{Прямые запросы}

\def\RPC{Musereum RPC}
\def\Nodes{Публичные узлы}
\def\Notaries{Нотариусы}
\def\EVM{Ethereum Virtual Machine}
\def\Blockchain{Blockchain реестр}
\def\IPFS{IPFS}
\def\ReadWrite{Чтение и запись}
\def\ReadOnly{Только чтение}

\begin{center}
\begin{tikzpicture}[node distance=1.5cm]
\node (DApps) 					[rect]
										{\DApps};
\node (Api)						[rect, right=0.2cm of DApps]
										{\Api};
\node (Requests)				[rect, right=0.2cm of Api]
										{\Requests};
\node (Interface)				[above=0.5cm of DApps.north west, anchor=west] {\Interface};										

\node (InterfaceRect)		[rect, rounded corners, dashed, fit={(Interface) (DApps) (Api) (Requests)}] {};

\path 
	let \p1=(DApps.west), \p2=(Requests.east)
	in node(RPC) [
		rect,
		below=1.5cm of DApps.south west,
		anchor=north west,
		minimum width=\x2-\x1-\pgflinewidth
	] {\RPC};
    
\path 
	let \p1=(RPC.west), \p2=(RPC.east)
	in node(Nodes) [
		rect,
		below=0.4cm of RPC.south west,
		anchor=north west,
		minimum width=(\x2-\x1-\pgflinewidth)
	] {\Nodes};
    
\path 
	let \p1=(Nodes.west), \p2=(Nodes.east)
	in node(Notaries) [
		rect,
		below=1.8cm of Nodes.south west,
		anchor=north west,
		minimum width=(\x2-\x1-\pgflinewidth)/2-0.3cm
	] {\Notaries};
    
\draw 
	let \p1=(Nodes.south), \p2=(Notaries.north) 
	in [arrow] (\x2,\y1) -- node [fill=white] {\ReadWrite} (Notaries.north);
	
\draw [arrow] (RPC) -- (Nodes);

\node (Connect)			[above=0.5cm of RPC.north west, anchor=west] {\Connect};
\node (ConnectRect)		[rect, rounded corners, dashed, fit={(RPC) (Nodes) (Notaries) (Connect)}] {};

\draw [thick] (Nodes) -- node [fill=white, yshift=15pt]{\ReadOnly} (ConnectRect.south);

\path
	let \p1=(Nodes.west), \p2=(Nodes.east)
	in node(EVM) [
		rect, 
		below=1.5cm of Notaries.south west,
		anchor=north west,
		minimum width=(\x2-\x1-\pgflinewidth)/2-0.1cm
	] {\EVM};
	
\path
	let \p1=(Nodes.west), \p2=(Nodes.east)
	in node(Blockchain) [
		rect, 
		below=0.4cm of EVM.south west,
		anchor=north west,
		minimum width=(\x2-\x1-\pgflinewidth)/2-0.1cm
	] {\Blockchain};
\path
	let \p1=(Nodes.west), \p2=(Nodes.east),
	     \p3=(EVM.north), \p4=(Blockchain.south)
	in node(IPFS) [
		rect, 
		below=0pt of EVM.north east,
		anchor=north west,
		xshift=0.2cm,
		minimum width=(\x2-\x1-\pgflinewidth)/2-0.1cm,
		minimum height=(-\y4+\y3)
	] {\IPFS};
\node (Underlayer)			[above=0.5cm of EVM.north west, anchor=west] {\Underlayer};
\node (UnderlayerRect)		[rect, rounded corners, dashed, fit={(EVM) (Blockchain) (IPFS) (Underlayer)}] {};
    
\draw [arrow] (EVM) -- (Blockchain);
\draw [arrow] (InterfaceRect) -- (ConnectRect);
\draw [arrow] (ConnectRect) -- (UnderlayerRect);
\end{tikzpicture} 
\end{center}

\subsection{Слой хранения и учета}
\label{tech-arch-underlayer}
\begin{multicols}{2}
Базовым слоем платформы является слой хранения и учета. Задача слоя предоставить возможность ведения децентрализованного реестра состояний («блокчейн») и децентрализованное хранения большого объема данных.

Каждый элемент слоя: блокчейн, хранилище, виртуальная машина, распределены по сети (узлам) и каждый желающий может создать свою, локальную копию. 
\end{multicols}
\subsection{Слой доступа}
\label{tech-arch-connect}
\begin{multicols}{2}
Задачи слоя доступа: предоставление, по запросу, информации из базового слоя и получение запросов об изменении.

Создание узла для организации дополнительной точки доступа никак не ограничивается платформой и доступно любому желающему. Данная особенность выгодно отличает платформу Musereum от других, централизованных музыкальных платформ:
\begin{itemize}
	\item Гарантирует 100\% up-time сети
	\item Устойчивость сети к цензуре
	\item Возможность заинтересованным сторонам создать собственную, локальную копию сети, а не доверять полученной информации от третьих лиц
\end{itemize}
\end{multicols}
\pagebreak
\subsection{Слой интерфейсов взаимодействия}
\label{tech-arch-interfaces}
\begin{multicols}{2}
Верхним слоем платформы является слой интерфейсов взаимодействия. Задача слоя получать пользовательский ввод и формировать из него понятные для протокола запросы. Запросы делятся на два типа: чтение и запись.

Платформа не накладывает каких-либо ограничений на чтение данных из блокчейна и децентрализованного хранилища. Единственное требование – формирования запроса в соответствии с соглашением об API.

Для записи (изменения состояния, сохранении данных) от клиента требуется предоставление валидной подписи запроса основанной на его уникальном приватном ключе и соблюдение других требований к записи (наличие прав, денежных средств для оплаты работы протокола и другие).

Для создания запроса не требуется организации собственной точки доступа (синхронизации узла с сетью) и в качестве клиента может выступать server-less приложение на web-технологиях \textit{(подробнее в \hyperref[tech-apps]{Децентрализованные приложения})}
\end{multicols}
\section{Блокчейн}
\label{tech-blockchain}
\begin{multicols}{2}
Протокол «Musereum» открытый, публичный, но требующий разрешения на производство блоков основанный на протоколе «Ethereum». Консенсус по единому состоянию сети в децентрализованной структуре производителей блоков достигается использованием алгоритма Proof-of-Authority (далее PoA). Для достижения консенсуса через PoA требуется:
\begin{enumerate}
\item Общеизвестный список нотариусов ($\mathbb{Y}$) с разрешением на производство блоков;
\item Контракт управления списком нотариусов ($\nu$).
\end{enumerate}
Изменение списка нотариусов (исключение, назначение нотариусов) изменяется контрактом управления по результатам голосования текущих нотариусов (большинством). 

\textit{Для первичного назначения составляется список из 12 нотариусов.}

Задача нотариуса: 
\begin{enumerate}
\item проверить поступившие транзакции участников системы,
\item собрать в единый блок изменения общего состояния,
\item заверять своей уникальной подписью собранный блок.
\end{enumerate}
Для соблюдения консенсуса, в протокол вводится внешняя конcтанта $\Gamma_{step}$, определяющая количество секунд в одном временном шаге или время между блоками. Musereum определяет константу $\Gamma_{step}$ как 5, или 1 блок раз в пять секунд.
\begin{equation}
\Gamma_{step} = 5
\end{equation}
По соглашению с алгоритмом консенсуса PoA, нотариус наделен правом создать один блок ($K$) за $a$ временных штампов $\Gamma_{time}$. Где $a$ равно количеству нотариусов:
\begin{equation}
a = |\mathbb{Y}|
\end{equation}
Выбор индекса нотариуса $i$ из множества $\mathbb{Y}$ для создания блока на штамп времени $b$ происходит по формуле: 
\begin{equation}
b = \frac{\Gamma_{time}}{\Gamma_{step}},  \quad b \in \mathbb{N} 
\end{equation}
\begin{equation}
i = b \ mod \ a
\end{equation}
\end{multicols}
\subsection{Выбор версии}
\label{tech-blockchain-score}
\begin{multicols}{2}
В случаях если нотариусы не могут прийти к общему (единому) состоянию системы и происходит «форк» (от англ. вилка) – появляются две или более цепочки блоков ($\mathbb{K}$), сеть определяет вес ($\beta_{score}(\mathbb{K}_c)$) каждой из цепочек исходя из количества нотариусов участвующих в создании блоков:
\begin{equation}
h = |\mathbb{K}_c| \quad h \ \text{is length of blockchain}
\end{equation}
\begin{equation}
\beta_{score}(\mathbb{K}_c) = \Gamma_{u128max} * h - m
\end{equation}
Сеть всегда выбирает цепочку с большим весом по $\beta_{score}$.
\end{multicols}
\subsection{Финализация блока}
\label{tech-blockchain-fin}
\begin{multicols}{2}
По консенсусу необратимым считается блок в цепочке после которого более 50\% нотариусов создали 2 или более блока.

Таким образом минимальное время полного подтверждения блока ($C$) будет:
\begin{equation}
C = 2 * \Gamma_{step} * |\mathbb{Y}|
\end{equation}
\textit{Или 120 секунд при изначальных настройках протокола: 12 нотариусов и блок раз в пять секунд}
\end{multicols}
\subsection{Свидетельство сделки}
\label{tech-blockchain-confirmation}
\begin{multicols}{2}
Для совершения любого действия требующего изменения состояния системы от пользователя требуется формирования транзакции ($T$) подписанной уникальной крипто-подписью ($T_{sign}$). 

Наличие крипто-подписи гарантирует сети, что транзакция была создана владельцем учетной записи и является изъявлением воли владельца.

Нотариус очередь которого формировать блок проверяет валидность транзакции и исполняет связанный код:
\begin{enumerate}
\item Отправитель имеет право на формирование транзакции к данному адресу – $\rho_{allowTx}(T_{from}, T_{to})$ (подробнее в \hyperref[tech-blockchain-rules]{Правила взаимодействия с EVM}		)
\item Транзакция соответствует формальным признакам –  $\rho_{initialValid}(T)$
\item Проверка подписи владельца учетной записи –  $\rho_{checkSign}(T)$
\item Наличие на счету ETM токенов для оплаты аванса работы вычислительной машины – $\rho_{upFront}(T)$
\item Выполнение связанного кода EVM не вызвало исключения – $\rho_{exception}(\rho_{execute}(T))$
\end{enumerate}
\end{multicols}
\begin{equation}
\begin{tabular}{ r l }
$\rho_{success}(T) =$ & $\rho_{allowTx}(T_{from}, T_{to}) > 0 \ \wedge $ \\
 \ & $\rho_{initialValid}(T) > 0 \ \wedge $ \\
 \ & $\rho_{checkSign}(T) > 0 \ \wedge $ \\
 \ & $\rho_{upFront}(T) > 0 \ \wedge $ \\
 \ & $\rho_{exception}(\rho_{execute}(T)) = 0 $
\end{tabular}
\end{equation}
\begin{multicols}{2}
Убедившись в валидности транзакции нотариус упаковывает транзакцию в блоки оповещает сеть о блоке и изменении состояния сети ($\mathbb{W}_h$).
\begin{equation}
\mathbb{W}_h = \rho_{state}(\mathbb{W}_{h-1}, \mathbb{K}_h)
\end{equation}
Другие нотариусы получают блок, проверяют ($\rho_{success}(T)$) и принимают решение о принятии его в цепочку. Создавая блок над $\mathbb{K}_h$ нотариус подтверждает его валидность и всех вложенных в него транзакций.

Таким образом количество заверяющих подписей у транзакции может быть от 1 до $|\mathbb{Y}|$.

Минимальное время получения $|\mathbb{Y}|$ подписей:
\begin{equation}
a = \Gamma_{step} * |\mathbb{Y}|
\end{equation}
\end{multicols}
\subsection{Вознаграждение за производство блоков}
\label{tech-blockchain-reward}
\begin{multicols}{2}
Вознаграждение за блок ($R$) состоит из двух частей: новой эмиссии и комиссий за совершенные действия.
 
В платформе Musereum эмиссия новых токенов ETM происходит в процессе создания блока. Размер эмиссии фиксирован – 3 ETM за каждый блок.

Размер комиссии за совершенные действия высчитывается для каждой транзакции отдельно. Отправитель указывает в транзакции стоимость единицы сложности ($T_{gasPrice}$) которую он готов потратить в качестве компенсации работы сети. 

Финальная комиссия высчитывается по формуле:
\begin{equation}
x = T_{gasPrice} * \rho_{dificulty}(\rho_{execute}(T))
\end{equation}

Общее вознагражение за блок:
\begin{equation}
\begin{aligned}
\text{Let } a \text{ is a list of transactions of block } \mathbb{K}_h \\
a = \mathbb{K}[h]_{transactions} \\
\mathbb{R}_h = 3 + \sum\limits_{i=0}^{|a|} a_i
\end{aligned}
\end{equation}
\end{multicols}

\subsubsection{Распределение вознаграждения}
\label{tech-blockchain-reward-distribution}
В отличие от протокола Ethereum и Bitcoin, в сети Musereum нету майнеров и основную ценность создают артисты, выбирающие Musereum платформой для выпуска и дистрибуции своих произведений.

Musereum высоко ценит вклад артистов в создание ценности и направляет большую часть награды за блок на субсидирование артистов.

Общая схема распределения:
\begin{itemize}
	\item 90\% – Musereum Foundation:
	\begin{itemize}
		\item 70\% – программы субсидирования артистов, например, \hyperref[tech-apps-soundchain-payperplay]{Pay-per-Play}
		\item 20\% – программа субсидирования аффилированных специалистов: регистраторы музыкальных активов, аудиторы смарт-контрактов
		\item 10\% – программы вознаграждения: Fans Bounty, Clipmaker Bounty и т.д.
	\end{itemize}
	\item 10\% – Производители блоков
\end{itemize}

% Nodes
\def\Notaries{Нотариусы}
\def\Musereum{Musereum Foundation}
\def\Artists{Артисты}
\def\Specialists{Специалисты}
\def\Bounties{«Баунти»}

\begin{center}
\begin{tikzpicture}[node distance=0pt]
	
\node(NotariesPart) [
		rect,
		anchor=west,
		minimum width=0.1*\textwidth,
		label=below:{\Notaries}
	] {};
	
\node(MusereumPart) [
		rect,
		right=0pt of NotariesPart.east,
		anchor=west,
		minimum width=0.9*\textwidth+1pt,
		minimum height=1.2cm,
		yshift=.1cm,
		label=above:{\Musereum}
	] {};
	
\node(ArtistsPart) [
		rect,
		below=0pt of NotariesPart.east,
		anchor=west,
		minimum width=0.9*0.7*\textwidth
	] {\Artists};
\node(SpecialistsPart) [
		rect,
		below=0pt of ArtistsPart.east,
		anchor=west,
		minimum width=0.9*0.2*\textwidth
	] {\Specialists};
\node(BountyPart) [
		rect,
		below=0pt of SpecialistsPart.east,
		anchor=west,
		minimum width=0.9*0.1*\textwidth,
		label=below:{\Bounties}
	] {};

\end{tikzpicture} 
\end{center}
\subsection{Виртуальная машина Ethereum}
\label{tech-blockchain-evm}
\subsection{Смарт-контракты Ethereum}
\label{tech-blockchain-contracts}

\subsection{Правила Musereum по взаимодействию с EVM}
\label{tech-blockchain-rules}
\begin{multicols}{2}
Для повышения безопасности участников платформы работа со смарт-контрактами EVM в Musereum отличается от таковой в родительской сети Ethereum.

Загрузить смарт-контракт в сеть может любой участник сети, но такой смарт-контракт получается статус $P_{development}$ и взаимодействовать с ним могут только аффилированные адреса учетных записей сети. Изначально это адрес пользовательской учетной записи с которой был загружен контракт.

Правила взаимодействия со смарт-контрактами регулируются смарт-контрактом разрешений: $\rho$.

Любой вызов смарт-контракта изначально запрашивает разрешение на совершение данного вызова – $\rho_{allowTx}(T_{from}, T_{to})$
\end{multicols}
\begin{equation}
a = \rho_{stateOf}(T_{to})
\end{equation}
\begin{equation}
b = \rho_{affilateWith}(T_{to})
\end{equation}
\begin{equation}
\rho_{allowTx}(T_{from}, T_{to}) = \begin{cases}1, & \text{if } a = P_{production} \\ 1, & \text{if } a = P_{development} \wedge T_{from} \in b \\ 0, & \text{overwise} \end{cases}
\end{equation}
Для получения статуса $P_{production}$ автор смарт-контракта обязан предоставить на проверку исходный код смарт-контракта и пройти процедуру проверки \textit{(подробнее в \hyperref[tech-apps-contracts-registry]{Реестр смарт-контрактов})}

% Nodes
\def\Owner{Владелец контракта}
\def\Blockchain{Блокчейн}
\def\Registry{Реестр контрактов}
\def\LoadByteCode{Загрузка байткода в блокчейн}
\def\LoadSourceCode{Загрузка исходного кода}
\def\ValidationEnded{Код прошел проверку?}
\def\ChangeStatus{Смена состояния смарт-контракта}

\begin{center}
\begin{tikzpicture}[node distance=1.5cm]
\node (LoadByteCode)		[rect, text width=11.5em] 
										{\LoadByteCode};

\node (LoadSourceCode)	[rect, text width=11.5em, below of=LoadByteCode] 
										{\LoadSourceCode};

\node (ValidationEnded)	[event, text width=11.5em, below=1.5cm of LoadSourceCode]
										{\ValidationEnded};
										
\node (aux0)[right=2cm of ValidationEnded]{};
\node (Invalid)					[text width=5em, above of=aux0]
										{$\pi_{ended} = 0$};

\node (ChangeStatus)		[rect, text width=11.5em, below of=ValidationEnded]	
										{\ChangeStatus};

\node (InDevelopment)		[inout, text width=11em, right=1cm of LoadByteCode]			
										{$\rho_{stateOf}(a) = P_{development}$};
										
\node (InProduction)			[inout, text width=11em, right=1cm of ChangeStatus]			
										{$\rho_{stateOf}(a) = P_{development}$};

%arrows
\draw [arrow] (LoadByteCode)				-- (LoadSourceCode);
\draw [-, thick, dashed]
	(LoadByteCode)								--(InDevelopment);
\draw [arrow] (LoadSourceCode) 			-- (ValidationEnded);
\draw [arrow] (ValidationEnded)			-- (ChangeStatus);
\draw [arrow] (ValidationEnded)			-| (Invalid);
\draw [thick]  (Invalid)  							-| (ValidationEnded);
\draw [-, thick, dashed] 
	(ChangeStatus)									-- (InProduction);

\end{tikzpicture} 
\end{center}

\section{Децентрализованное хранилище}
\label{tech-storage}
\subsection{Обзор технологии}
\label{tech-storage-review}
\subsection{Соглашение о хранении данных}
\label{tech-storage-convention}
\subsection{Система удобочитаемых наименований}
\label{tech-storage-naming}


\section{Децентрализованные приложения}
\label{tech-apps}
Децентрализованные приложения Musereum реализуют код для слоя учета и хранения и код для интерфейса взаимодействия. Протокол позволяет взаимодействовать приложениям между собой.

\subsection{Governance dApp}
\label{tech-apps-governance}
\begin{multicols}{2}
Децентрализованное приложение управления списком нотариусов и проверки их прав на производство блоков.
\end{multicols}

\subsection{Contract Registry}
\label{tech-apps-contracts-registry}
\begin{multicols}{2}
Децентрализованное приложение для регистрации смарт-контрактов платформы и их учета. 

Смарт-контракт приложения позволяет получить:
\begin{enumerate}
	\item Список всех зарегистрированных смарт-контрактов
	\item Получить состояние контракта ассоциированное с адресом:
	\begin{enumerate}
		\item \code{Unknown} – неизвестное состояние (по умолчанию для неизвестных для Contract Registry адресов)
		\item \code{Development} – смарт-контракт зарегистрирован, но находится в разработке (взаимодействие доступно только аффилированным пользователям)
		\item \code{Production} – смарт-контракт зарегистрирован и доступен для взаимодействия всем желающим
		\item \code{Halt} – смарт-контракт зарегистрирован, но отключен (по решению автора) 
		\item \code{Suspend} – смарт-контракт зарегистрирован, но отключен (по решению аудиторов)
	\end{enumerate}
\item Получить ссылку на репозиторий с исходным кодом
\item Получить ссылку на ABI JSON для взаимодействия со смарт-контрактом
\item Получить список аффилированных аудиторов
\item Получить экземпляр голосования \textit{(подробнее в \hyperref[tech-apps-voting]{VotingdApp})} для принятия решения о смене статуса
\end{enumerate}
\end{multicols}
\subsubsection{Исходный код и проверка смарт-контракта}
\label{tech-apps-contracts-validate}
\begin{multicols}{2}
Валидация смарт-контрактов происходит в трех направлениях:
\begin{itemize}
	\item Валидация исходного кода на:
		\begin{enumerate}
			\item Наличие уязвимостей
			\item Соответствие заявленной бизнес-логики
			\item Отсутствие алгоритмических ошибок
		\end{enumerate}
	\item Соответствие праву юрисдикций в которых работает платформа Musereum
	\item Соответствие концепции Musereum
\end{itemize}
\vfill\null
\columnbreak
Для одобрения смарт-контракта необходимо единогласное согласие от всех аффилированных аудиторов по трем направлениям. 

Отслеживать ход валидации можно в \hyperref[tech-apps-voting]{Voting dApp} в соответсвующей \code{ContractValidationBallot}.
\end{multicols}
\subsection{Musereum Name System (MNS)}
\label{tech-apps-mns}
\begin{multicols}{2}
Реестр ассоциированных имен к адресам в платформе Musereum? требуется для предоставления доступа к актуальным версиям приложений протокола без необходимости запоминать внутренний адрес (20 символьный hex-адрес).

Запись об имени может создать любой пользователь сети для произвольного адреса. Количество имен для адреса никак не ограничено. 

Для защиты от нецелевого использования имен, за регистрацию нового имени взимается плата в ETM равная текущей стоимости.

Запись об имени является производным от MNS смарт-контрактом: \code{NameRecordContract}.

Смарт-контракт содержит информацию о владельце имени и ассоциированном адресе в сети и предоставляет возможности:

Смены владельцем ассоциированного адреса – 
Смены владельцем владельца (уступление прав) – 
MNS накладывает ограничения на доступные имена:
Длина имени не может превышать 20 символов (или менее при non-ASCII символах)
Длина имени не может быть менее 4-х символов
\end{multicols}
\subsubsection{Интерфейс взаимодействия}
\label{tech-apps-mns-api}
\code{mapping (address => address) names} – ($\eta_{names}$)\hfill\null\linebreak
Ассоциативный список адрессов к адрессам смарт-контрактов регистрации имени

\code{uint price} – ($\eta_{price}$)\hfill\null\linebreak
Стоимость регистрации имени в токенах ETM

\code{event NewName(address indexed name, byte20 ascii)} – ($\eta_{NewName}$)\hfill\null\linebreak
Журнал событий регистрации новых имен

\code{event AssociateWith(address indexed name, address indexed target)} – ($\eta_{AssociateWith}$)\hfill\null\linebreak
Журнал событий ассоциирования адреса со смарт-контрактом регистрации имени

\code{function buyName(byte20 acsii) payable} – ($\eta_{buy}$)\hfill\null\linebreak
Метод покупки имени. Возвращает \code{address} если покупка прошла успешно и \code{throw} при ошибке.
\begin{equation}
\eta_{buy}(s) = \begin{cases}
	throw, & \text{if } s \in \eta_{NewName} \\ 
	throw, & \text{if } |s| < 4 \\
	throw, & \text{if } T_{value} < \eta_{price} \\
	address, & \text{overwise}
\end{cases}
\end{equation}

\paragraph{Смарт-контракт NameRecord}– ($\eta'$)

\code{address owner} – ($\eta'_{owner}$)\hfill\null\linebreak
Текущий владелец имени

\code{byte20 ascii} – ($\eta'_{ascii}$)\hfill\null\linebreak
Зарегистрированное имя

\code{address target} – ($\eta'_{target}$)\hfill\null\linebreak
Текущая ассоциация с адресом на платформе Musereum

\code{function associate(address target)} – ($\eta'_{associate}$)\hfill\null\linebreak
Изменить ассоциацию имени на новый адрес. Возвращает \code{true} если прошло успешно и \code{throw} при ошибке.
\begin{equation}
\eta'_{associate}(a) = \begin{cases}
	throw, & \text{if } T_{from} \neq \eta'_{owner} \\
	throw, & \text{if } a = 0 \\
	true, & \text{overwise}
\end{cases}
\end{equation}

\subsection{Voting dApp}
\label{tech-apps-voting}
\begin{multicols}{2}
Приложение децентрализованного голосования состоит из связанных элементов:
\begin{itemize}
	\item \code{BallotsManager} ($\phi$)\hfill\null\linebreak
	реестр голосований
	\item \code{Ballot} ($\pi$)\hfill\null\linebreak
	абстрактный смарт-контракт голосования
	\item \code{VotingRightsToken} ($\phi^\tau$) \hfill\null\linebreak
	специализированный токен выдаваемый участникам голосования для доказательства своих прав на участие в голосовании
	\item Интерфейса взаимодействия с голосованиями
\end{itemize}
\vfill\null\columnbreak
Для создания голосования инициатор предложения должен сформировать валидный \code{Ballot} – смарт-контракт ведущий учет голосов и реализующий \code{applyProposal} метод для изменения состояния в соответствии с предложением. 
\end{multicols}

\begin{center}
\begin{tikzpicture}[node distance=1.5cm]
\node (LoadByteCode)		[rect, text width=11.5em] 
										{\LoadByteCode};

\node (LoadSourceCode)	[rect, text width=11.5em, below of=LoadByteCode] 
										{\LoadSourceCode};

\node (ValidationEnded)	[event, text width=11.5em, below=1.5cm of LoadSourceCode]
										{\ValidationEnded};
										
\node (aux0)[right=2cm of ValidationEnded]{};
\node (Invalid)					[text width=5em, above of=aux0]
										{$\pi_{ended} = 0$};

\node (ChangeStatus)		[rect, text width=11.5em, below of=ValidationEnded]	
										{\ChangeStatus};

\node (InDevelopment)		[inout, text width=11em, right=1cm of LoadByteCode]			
										{$\rho_{stateOf}(a) = P_{development}$};
										
\node (InProduction)			[inout, text width=11em, right=1cm of ChangeStatus]			
										{$\rho_{stateOf}(a) = P_{development}$};

%arrows
\draw [arrow] (LoadByteCode)				-- (LoadSourceCode);
\draw [arrow] (LoadByteCode)				-- (InDevelopment);
\draw [arrow] (LoadSourceCode) 			-- (ValidationEnded);
\draw [arrow] (ValidationEnded)			-- (ChangeStatus);
\draw [arrow] (ValidationEnded)			-| (Invalid);
\draw [thick]  (Invalid)  							-| (ValidationEnded);
\draw [arrow] (ChangeStatus)				-- (InProduction);

\end{tikzpicture} 
\end{center}

\subsubsection{Смарт-контракт Ballot}
\label{tech-apps-voting-ballot}
\begin{multicols}{2}
Абстрактный смарт-контракт описывающий требования к финальной реализации предложений для голосования.

\code{address votingToken} – ($\pi_{token}$)\hfill\null\linebreak
Ассоциированный с голосованием токен наделяющий держателя правом голоса

\code{function vote(bool)} – ($\pi_{vote}$)\hfill\null\linebreak
Внешний метод для записи решения, вызывающего метод.

\code{function applyProposal()} – ($\pi_{apply}$)\hfill\null\linebreak
Абстрактный метод применения предложения по факту успешного голосования. Реализуется в дочерних смарт-контрактах.

\code{uint agreeVotesCount} – ($\pi_{agree}$)\hfill\null\linebreak
Количество токенов переданных для принятия решения.

\code{uint rejectVotesCount} – ($\pi_{reject}$)\hfill\null\linebreak
Количество токенов переданных для отклонения решения.

\code{uint endTime} – ($\pi_{end}$)\hfill\null\linebreak
Время завершения голосования.

\code{function isEnded()}  - ($\pi_{ended}$)\hfill\null\linebreak
Возвращает $1$ если время для голосования закончилось и $0$ в других случаях.
\begin{equation}
\pi_{ended} = \begin{cases}
	1, & \text{if } \pi_{end} > \Gamma_{time} \\
	0, & \text{overwise}
\end{cases}
\end{equation}

\code{function voteMajorityRule()} – ($\pi_{rule}$)\hfill\null\linebreak
Результат функции $f(x) = 1/x$ где $x$ это отношение голосов за принятие к общему количеству голосов достаточное для принятие решения:
\begin{equation}
\pi_{rule} = \begin{cases}
	\infty, & \text{if } x = 0 \\
	\frac{1}{x}, & \text{overwise}
\end{cases}
\end{equation}

\code{function compare()} – ($\pi_{compare}$)\hfill\null\linebreak
Возвращает 1 если количество голосов за принятие решение удовлетворяет условия голосования для принятия решения.
\begin{equation}
\pi_{compare} = \begin{cases}
	1, & \text{if }\pi_{agree} * \pi_{rule} > \pi_{agree} + \pi_{reject} \\
	0, & \text{overwise}
\end{cases}
\end{equation}

\code{function isSuccess()} – ($\pi_{success}$)\hfill\null\linebreak
Возвращает 1 если голосование успешно завершено и 0 в других случаях.
\begin{equation}
\pi_{success} = \begin{cases}
	1, & \text{if } \pi_{ended} \wedge \pi_{compare} \\
	0, & \text{overwise }
\end{cases}
\end{equation}
\end{multicols}
\subsection{Musical Assets Registry}
\label{tech-apps-assets}
\begin{multicols}{2}
Приложение регистрации музыкальных активов на платформе Musereum. Приложение состоит из:
\begin{enumerate}
	\item \code{MusicalAssetsRegistry} (MAR – $\psi$)\hfill\null\linebreak
	Корневой смарт-контракт приложения. Выполняет функции фабрики и реестра.
	\item \code{AssetContract} – ($\alpha$)\hfill\null\linebreak
	Производимые в MAR смарт-контракты музыкальных активов.
	\item Ассоциированные с \code{AssetContract}'ами смарт-контракты \hyperref[tech-apps-dal]{децентрализованных лейблов}
	\item \code{AssetRegistryBallot} – ($\pi^{regAsset}$)
	\item \code{AssetUnregistryBallot} – ($\pi^{unregAsset}$)
	\item Интерфейса взаимодействия – веб-приложения в рамках Musereum Wallet
\end{enumerate}
\end{multicols}
\subsubsection{Смарт-контракт MusicalAssetsRegistry}
\label{tech-apps-assets-registry}
\begin{multicols}{2}
Задачи смарт-контракты: учет зарегистрированных музыкальных активов, создание заявок на регистрацию музыкального актива через голосование в \hyperref[tech-apps-voting]{Voting dApp}.

Голосование проходит среди авторизованных системой регистраторов. 

Реестр полностью публичный и предоставляет возможности для аудита данных всем участникам сети. 

Получение исторических данных возможно через создание запроса к контракту за журналом событий:
\end{multicols}

\code{event NewAsset(address indexed asset, uint indexed index)} – ($\psi_{NewAsset}$)\hfill\null\linebreak
Возвращает журнал событий добавления музыкальных активов в реестр с присвоенными им индексами в реестре. 
\textit{Событие создается в момент создание заявки на регистрацию, а не при прохождение регистрации.}
\begin{equation}
\psi_{NewAsset} = \mathbb{E}[asset, index], \quad a \in \mathbb{A} \wedge i \in \mathbb{N}
\end{equation}

\code{event RegistryAsset(address indexed asset, uint indexed index)} – ($\psi_{RegistryAsset}$)\hfill\null\linebreak
Возвращает подмножество журнала событий $\psi_{NewAsset}$ ассоциированных к адресам активов успешно прошедших голосование за добавление в публичный реестр.
\begin{equation}
\psi_{RegistryAsset} \subset \psi_{NewAsset}
\end{equation}

\code{event UnregistryAsset(address indexed asset, uint indexed index)} – ($\psi_{UnregistryAsset}$)\hfill\null\linebreak
Возвращает подмножество $\psi_{RegistryAsset}$ для музыкальных активов удаленных из публичного реестра Musereum по решению сообщества.
\begin{equation}
\psi_{UnregistryAsset} \subset \psi_{RegistryAsset} \subset \psi_{NewAsset}
\end{equation}

\code{mapping (address => uint) assets} – ($\psi_{assets}(a)$)\hfill\null\linebreak
Ассоциативный словарь добавленных в реестр музыкальных активов.
\begin{equation}
\psi_{assets}(a) = \begin{cases}
	i, & \text{if } a \in \mathcal{S}(\psi_{NewAsset}, [{asset}]) \\
	0, & \text{overwise}
\end{cases}, \quad a \in \mathbb{A} \wedge i \in \mathbb{N}_1
\end{equation}

\code{address[] public assetsIndex} – ($\psi_{index}(i)$)\hfill\null\linebreak
Список всех добавленных в реестр музыкальных активов. Индекс актива в списке равен индексу в ассоциативном словаре минус 1.
\begin{equation}
i = \psi_{assets}(a), \quad a \in \mathbb{A} \wedge i \in \mathbb{N}
\end{equation}
\begin{equation}
\psi_{index}(i - 1) \equiv a, \quad \text{if } i > 0
\end{equation}

\subsubsection{Смарт-контракт AssetContract}
\label{tech-apps-assets-contract}
Запись в блокчейне о зарегистрированном музыкальном активе. Содержит всю необходимую информацию для идентификации музыкального актива и определения природы актива.

\code{string name} – ($\alpha_{name}$)\hfill\null\linebreak
Ассоциированное с активом имя

\code{Multihash meta} – ($\alpha_{meta}$)\hfill\null\linebreak
Хеш ассоциированной с активом мета-информацией (хранится в децентрализованном хранилище)
\begin{equation}
\alpha_{meta} = \mathcal{O}[hash, hashFunction, size],
\end{equation}
\begin{equation}
\alpha_{meta}[hash] \in \mathbb{B}_{32} \wedge \alpha_{meta}[hashFunction] \in \mathbb{B} \wedge \alpha_{meta}[size] \in \mathbb{B}
\end{equation}

\code{event SetAssetType(uint8 indexed typeId)} – ($\alpha_{SetAssetType}$)\hfill\null\linebreak
Возвращает журнал событий определения типа актива, содержащее 	индекс перечисления типом активов (\code{AssetTypes} – $\alpha^{types}$).
\begin{equation}
\alpha^{types} \in \mathbb{B}
\end{equation}
\begin{equation}
\alpha_{SetAssetType} = \mathbb{E}[typeId], \quad i \in \alpha^{types}
\end{equation}

\code{mapping (uint8 => bool) type} – ($\alpha_{type}(i)$)\hfill\null\linebreak
Ассоциативный словарь типов актива.
\begin{equation}
\alpha_{type}(i) = \begin{cases}
1, & \text{if } i \in \alpha_{SetAssetType} \\
0, & \text{overwise}
\end{cases}, \quad i \in \alpha^{types}
\end{equation}

\subsubsection{Смарт-контракт AssetRegistryBallot}
\label{tech-apps-assets-regballot}
Реализация Ballot смарт-контракта для проведения голосования за одобрение актива в реестре Musereum.
\subsubsection{Смарт-контракт AssetUnregistryBallot}
\label{tech-apps-assets-unregballot}
Реализация Ballot смарт-контракта для проведения голосования об исключении актива из реестре Musereum.
\subsection{Decentralized Autonomous Labels}
\label{tech-apps-dal}
\begin{multicols}{2}
Децентрализованный автономный лейбл – это смарт-контракт управления ассоциированным с ним музыкальным активом. В возможности DAL входит:
\begin{enumerate}
	\item Определение устава лейбла регулирующего правила принятия решений;
	\item Выпуск и продажа лицензий на коммерческое использование связанного музыкального актива;
	\item Аккумуляция и распределение дохода музыкального актива,
	\item Распределение и учет прав на музыкальный актив с цифровым доказательство – токеном.
\end{enumerate}

Управление музыкальным активом происходит по принципу демократического голосования держателями токенов. 

Возможности DAL реализует набор смарт-контрактов:
\begin{itemize}
	\item\code{DALRegistry} – ($\delta^R$)\hfill\null\linebreak
	Реестр зарегистрированных децентрализованных автономных лейблов
	\item\code{DALContract} – ($\delta$)\hfill\null\linebreak
	Cмарт-контракт децентрализованного автономного лейбла
	\item\code{DALToken} – ($\tau^\delta$)\hfill\null\linebreak
	Смарт-контракт токена подтверждающего права держателя на участие в децентрализованном лейбле
	\item Набор смарт-контрактов голосования реализации Ballot:
	\begin{itemize}
		\item\code{CharterChangeVotingRulesBallot}
		Предложение по голосованию за смену правил голосования о внесении изменений в устав
		\item\code{CharterChangeProposalBallot}\hfill\null\linebreak
		Предложение по внесению изменений в устав
		\item\code{AddLicenseBallot}\hfill\null\linebreak
		Предложение по созданию новой лицензии
		\item\code{CloseLicenseBallot}\hfill\null\linebreak
		Предложение о прекращении продажи лицензии
		\item\code{ReplaceLicenseBallot}\hfill\null\linebreak
		Предложение о смене/внесению изменений в лицензию
	\end{itemize}
\item\code{DALAssetLicense} – ($\theta$)\hfill\null\linebreak
Смарт-контракт лицензии на коммерческое использование
\item\code{DALAssetLicenseRule} – ($\theta'$)\hfill\null\linebreak
Производный от лицензии смарт-контракт ограничения использования
\end{itemize}
\end{multicols}
\vfill\null\pagebreak
\subsubsection{Смарт-контракт DALRegistry}
\label{tech-apps-dal-registry}
Реестр зарегистрированных автономных лейблов необходим для учета и доступа к лейблу управляющему ассоциированным музыкальным активом.

\code{mapping (address => address) assetLabels} – ($\delta^R_{labels}(a)$)\hfill\null\linebreak
Ассоциативный словарь созданных децентрализованных лейблов.

\code{event CreateLabel(address indexed asset,}\hfill\null\linebreak
\code{~~~~~~~~~~~~~~~~~~address indexed label,}\hfill\null\linebreak
\code{~~~~~~~~~~~~~~~~~~uint indexed index)~~~} – ($\delta^R_{CreateLabel}$)\hfill\null\linebreak
Возвращает множество событий создания децентрализованных лейблов. Событие включает в себя: адрес созданного лейбла, адресом ассоциированного музыкального актива и присвоенный лейблу индекс в реестре.
\begin{equation}
\begin{aligned}
\delta^R_{CreateLabel} \in \mathcal{E}(asset, label, index) \\
asset \in \mathcal{S}(\psi_{RegistryAsset}, [asset]) \subset \mathbb{A}^\alpha \ \wedge \\
label \in \mathbb{A}^\delta \ \wedge \\
index \in \mathbb{N}_1
\end{aligned}
\end{equation}
\code{event CreateLabelBallot(address indexed label,~} \hfill\null\linebreak
\code{~~~~~~~~~~~~~~~~~~~~~~~~address indexed ballot)} – ($\delta^R_{CreateLabelBallot}$)\hfill\null\linebreak
Возвращает множество событий создания голосований децентрализованными лейблами. 

Событие включает в себя: адреса децентрализованного лейбла и адрес созданного голосования.
\begin{equation}
\begin{aligned}
\delta^R_{CreateLabelBallot} \in \mathbb{E}[label, ballot] \\
label \in \mathcal{S}(\delta^R_{CreateLabel}, [label]) \subset \mathbb{A}^\delta \ \wedge \\
ballot \in \mathcal{S}(\phi_{CreatedBallot}, [ballot]) \subset \mathbb{A}^\pi
\end{aligned}
\end{equation}
\code{event FinishLabelVoting(address indexed label,~}\hfill\null\linebreak
\code{~~~~~~~~~~~~~~~~~~~~~~~~address indexed ballot,}\hfill\null\linebreak
\code{~~~~~~~~~~~~~~~~~~~~~~~~bool indexed result)~~~} – ($\delta^R_{FinishLabelVoting}$)\hfill\null\linebreak
Возвращает подмножество $\delta^R_{CreateBallot}$ для завершенных  голосований с явным указанием успеха или провала предложения.
\begin{equation}
\begin{aligned}
f_Z(\mathbb{E}') = \{\mathbb{E}'_i : \ \forall \ \mathbb{E}'_i[ballot][ended] > 0 \} \\
\delta^R_{FinishLabelVoting} = f_Z(\delta^R_{CreateLabelBallot}) \\
\end{aligned}
\end{equation}

\code{event NewLabelLicense(address indexed label, address indexed license)} – ($\delta^R_{NewLicense}$)\hfill\null\linebreak
Возвращает множество событий созданных децентрализованным лейблом лицензий.

\code{event RemoveLabelLicense(address indexed label, address indexed license)} – ($\delta^R_{RemoveLicense}$)\hfill\null\linebreak
Возвращает подмножество $\delta^R_{NewLicense}$ для снятых с продажи лицензий.

\subsubsection{Смарт-контракт DALContract}
\label{tech-apps-dal-label}
\begin{multicols}{2}
Смарт-контракт управления музыкальным активом. Управление активом и связанными сущностями происходит путем вынесения предложения на голосование с предварительным распределением \code{VotingRightsToken} всем участникам сообщества в соответствии с балансом связанного токена \code{DALToken}.
 
Децентрализованный лейбл создается автоматически по факту регистрации связанного актива в реестре \code{MusicalAssetsRegistry}.

При создании децентрализованного лейбла выпускается 1,000,000,000 токенов подтверждения прав, для упрощения расчетов в связанных пользовательских интерфейсах фактическое число делится на $10^7$. 

Таким образом единоличный контроль лейблом эквивалентен 100.0000000 токенам, а минимальный голос в компании – 0.0000001\%.

Права держателей токенов \code{DALToken}:
\begin{enumerate}
	\item Участие в распределении роялти
	\item Участие в голосовании за принятие решений 
	\item Уступление токенов другим участником платформы через обмен на иные токены или ETM
\end{enumerate}
\end{multicols}
\subsubsection{Смарт-контракт DALToken}
\label{tech-apps-dal-token}
\begin{multicols}{2}
Токен децентрализованного лейбла является \hyperref[tech-blockchain-contracts]{ERC-20} совместимым токеном.

Для обеспечения работы системы распределения роялти в код стандартного ERC-20 токена добавленны необходимые изменения:
\begin{enumerate}
	\item Сохраняется one-to-one связь с ассоциированным лейблом в поле \code{address label}\vfill\null\columnbreak
	\item Внесены требуемые для \hyperref[tech-apps-dal-royalty-optimization]{оптимизации алгоритма распределения} вызовы \code{withdrawRevenueFor()} в \code{transfer(...)} и \code{transferFrom(...)} функции для отправителя и получателя
	\item Реализован метод \code{makeVoteToken()} для создания \code{VotingRightsToken}'а соответствующего текущим балансам
\end{enumerate}
\end{multicols}
\subsubsection{Распределение роялти}
\label{tech-apps-dal-royalty}
\begin{multicols}{2}
Роялти – полученные в ходе коммерческой активности связанного музыкального актива единицы ETM на баланс связанного децентрализованного автономного лейбла.

Каждый участник лейбла в любой момент времени может реализовать свое право на получение доли роялти в соответствии с текущим количество токенов.
\end{multicols}
\begin{equation}
\def\arraystretch{1.3}
\begin{array}{@{}llll@{}}
\toprule
    & \multicolumn{2}{c@{}}{\text{Распределение роялти}} & \\
\cmidrule(l){2-3}
    & a = T_{from} & \text{Initial withdrawal transaction} & \\
    & d = \delta(T_{to}) & \text{DAL instance} & \\
    & r = \tau^\delta(d_{token}) & \text{Rights tokens of corresponding DAL} & \\
    & h = \mathbb{K}_n & \text{Height of latest block} & \\
    & b_h = r_{balanceOf}(a) & \text{Balance of DAL rights tokens at block } h & \\
    & t_h = r_{totalSupply} & \text{Total supply of rights token (typicaly is } 10^9 \text{)} & \\
    & s_h = \frac{b_h}{t_h} & \text{Stake of } a \text{ at moment } h & \\
    & I_h = x & I_h \text{ Income of a DAL at moment } h & \\
    & T_h = \sum\limits^{h}_{i=1} I_i & \text{Total lifetime income of DAL} & \\
    & W_h = s * (T_h - \sum\limits^{h-1}_{i=1} W_i) & \text{Available withdrawal amount at moment } h & \\
\bottomrule
\end{array}
\end{equation}
\subsubsection{Оптимизация алгоритма распределения}
\label{tech-apps-dal-royalty-optimization}
\begin{multicols}{2}
В целях оптимизации и устранения необходимости считать $W_h$ для каждого уникального $h$, в смарт-контракте вводится вспомогательный ассоциативный список \code{incomeAtLatestWithdraw(address => uint)}, сохраняющий значение в момент последнего вывода средств для ассоциированного адреса. 

Такой подход позволяет производить расчет роялти исключительно в момент когда этого требует бизнес-логика и только для связанных с ней держателей. 

При уступлении прав третьему держателю (перевод токенов) автоматически запускается алгоритм расчета доступного роялти для текущего и будущего держателей.
\end{multicols}
\subsubsection{Голосование за принятие решений}
\label{tech-apps-dal-voting}
\subsubsection{Устал децентрализованного лейбла}
\label{tech-apps-dal-charter}
\subsubsection{Продажа лицензий и смарт-контракт DALAssetLicense}
\label{tech-apps-dal-license}
\begin{multicols}{2}
%Лицензии на базе смарт-контрактов привязываются к токенам трека и определяют характер использования композиции и ценовую политику, включая различные ценовые настройки в зависимости от географии использования. Коммерческие лицензии обладают сроком действия, задаваемым администратором. Перевыпуск лицензии по окончанию срока создается автоматически с теми же параметрами и отправляется запрос/уведомление акционерам трека. Если акционеры в течение определенного периода выразили свое несогласие, то администратор перевыпускает лицензии с другими параметрами. Консенсус достигается при согласии 51% держателей токенов на трек. Смарт-лицензия состоит из четырех элементов: юридического и простого текста, кода смарт-контракта, механизма API для чтения метаданных и встраивания продаж на сторонние платформы. Смарт-лицензии распределяют роялти между держателями токенов трека. 

Лицензирование “Musereum” включает в себя возможность проведения следующих типов лицензирования/

Продажа лицензий на коммерческое использование связанного музыкального актива основная задача децентрализованного лейбла, а получение роялти за продажу основная мотивация существования.

Все лицензии выпускаются в виде экземпляря смарт-контракта \code{DALAssetLicense} через интерфей усправления \code{DALContract}. 

Задача корневого контракта \code{DALAssetLicense} определить природу лицензии, правила приобретения и ограничения на характер использования. \code{DALAssetLicense} является контейнером и предоставляет держателю неограниченную лицензию на коммерческое использование, ограничения же описываются в прозводных смарт-контрактах: \code{DALAssetLicenseRule}.

\end{multicols}
\paragraph{DALAssetLicenseRule}\hfill\null\linebreak
\begin{multicols}{2}
Абстрактный смарт-контракт, определяющий ограничение родительской лицензии, существует 7 базовых типов ограничений:
\begin{enumerate}
	\item \code{EnumerateRule} – органичение перечислением
	\item \code{RangeRule} – ограничение диапазоном
	\item \code{ValueRule} – ограничение значением
	\item \code{AddressRule} – ограничение адрессом(ами)
	\item \code{TextRule} – описательное ограничение 
	\item \code{AllOfRule} – группирующее ограничение по типу: все из группы
	\item \code{AnyOfRule} – группирующее ограничение по типу: любое из группы
\end{enumerate}

Финальная реализация ограничения может быть выбрана из предзаготовленных или создана под конкретную лицензию децентрализованным лейблом.
\end{multicols}

\textbf{Сравнение ограничений}
\begin{equation}
\begin{aligned}
a = \theta'(original), \quad b = \theta'(target) \\
\theta'_{fitWith}(b) = \begin{cases}
	1, & \text{if } a_{type} = b_{type} \ \wedge \ f_z(a, b) \\
	0, & \text{overwise}
\end{cases}
\end{aligned}
\end{equation}
\code{EnumerateRule}
\begin{equation}
f_z(a, b) = \begin{cases}
	a_{items} \subset b_{items}, & \text{if } a_{all} \\
	|\{i: \forall \ a_{items}[i] \in b_{items}\}|> 0, & \text{overwise}
\end{cases}
\end{equation}
\code{RangeRule}
\begin{equation}
\begin{aligned}
f_z(a, b) = a_{min} > b_{min} \ \wedge \ b_{max} > a_{max}
\end{aligned}
\end{equation}
\code{ValueRule}
\begin{equation}
\begin{aligned}
f_z(a, b) = \begin{cases}
	a_{value} = b_{value}, & \text{if } a_{equal} \\
	a_{value} < b_{value}, & \text{if } a_{less} \\
	a_{value} > b_{value}, & \text{if } a_{more}
\end{cases}
\end{aligned}
\end{equation}
\code{AddressRule}
\begin{equation}
f_z(a, b) = a_{addresses} \in b_{addresses}
\end{equation}
\code{AllOfRule}
\begin{equation}
f_z(a, b) = ||\{\top: a_{items}[i]_{fitWith}(b) = \top\}|| = ||a||
\end{equation}
\code{AnyOfRule}
\begin{equation}
f_z(a, b) = ||\{\top: a_{items}[i]_{fitWith}(b) = \top\}|| > 0
\end{equation}


\begin{table}[h]
\centering
\caption{Предзаготовленные типовые ограничения лицензий}
\begin{tabular}{p{0.2\linewidth}p{0.65\linewidth}cc}
\toprule
Наименование & Описание \\
\bottomrule
\toprule
\multicolumn{2}{c}{\code{EnumerateRule}} \\
\midrule
	\code{LocationRule} & Ограничение по географии использования и производные: \code{CountryRule} и \code{CityRule} \\
	\code{ChannelRule} & Ограничение по канналу воспроизведения \\
	\code{MediaRule} & Ограничение по типу носителя \\
	\code{GenreRule} & Ограничение по жанру носителя (используется для кино, театра и т.д.) \\
	\code{UsageRule} & Ограничение по типу использования \\
\bottomrule
\toprule
\multicolumn{2}{c}{\code{RangeRule}} \\
\midrule
	\code{PlayTimesRule} & Ограничение количеству/длительности \\ воспроизведения \\
	\code{LifetimeRule} & Ограничение срока действия лицензии \\
	\code{NumberOfCopyRule} & Ограничение по количеству копий \\
	\code{PowerOfRule} & Ограничение по бюджету проекта и(или) доходу компании приобретателя \\
\bottomrule
\toprule
\multicolumn{2}{c}{\code{ValueRule}} \\
\midrule
	\code{PriceRule} & Стоимость лицензии \\
\bottomrule
\end{tabular}
\end{table}

\subsubsection{Устав децентрализованного лейбла}
\label{tech-apps-dal-charter}
%\subsection{Fundraising dApp}
%\label{tech-apps-fundraising}
\begin{multicols}{2}
Любой \hyperref[tech-apps-dal]{децентрализованный лейбл} 
\end{multicols}
\subsection{Soundchain}
\label{tech-apps-soundchain}
\begin{multicols}{2}
Soundchain является первым проектом направленным на освоение \hyperref[tech-blockchain-reward]{Musereum Foundation}. Основная задача Soundchain начисление роялти децентрализованным лейблам за прослушивание музыкальных активов. 

Работу Soundchain определяет множество смарт-контрактов и интерфейсов взаимодействия, рассмотрим смарт-контракт PayPerPlay через который:
\begin{itemize}
	\item учитывается каждое прослушивание музыкального актива
	\item генерируется список выплат роялти из выделенного объема Musereum Foundation
\end{itemize}
\end{multicols}
\begin{equation}
\end{equation}
\begin{align*}
& \text {Let } n \text{ is a moment of time and} \\
& \text{Let } r \text{ is a money available to pay-per-play for} \\
& \text{Set of plays } a \text{ , where } a_i \text{ is a tuple:}  \\
a = &\{a_i \ | \ \forall i \in a, a_i \equiv \mathcal{O}[listener, asset, times, sign] \} \\
b = &\{b_i \ | \ \forall i \in a, b_i = f_{pay}(a_i), b_i \in (0; 1) \} \\ 
c = &\sum\limits^{|b|}_{i=0} b_i \\
d = &\frac{r}{c} \\
& \\
& \text{Finaly } p \text{ is a set of payments: } \\
p = &\{ p_i \ | \ \forall i \in b, p_i = d * b_i \} \\
r = &\sum\limits^{|p|}_{i=0} p_i
\end{align*}

Остается определить, что такое функция $f_{pay}$. Результат функции зависит от двух переменных: коэффициента слушателя и коэффициента лейбла: 

\begin{equation}
\end{equation}
\begin{align*}
f_{pay}(a) = min(&f_{listenerCoef}(a)),\\
						 &f_{labelCoef}(\delta^{R}_{labels}(a_{asset}))
\end{align*}

Коэффициет слушателя определяет насколько платформа доверяет данному слушателю:
\begin{enumerate}
	\item Персонификация учетной записи: ФИО, География;
	\item Подтверждение телефонного номера (Proof-of-SMS);
	\item Подтверждение географического происхождения: верификация документов;
	\item Размер материальной заинтересованности в будущем платформы: количество ETM, количество и объем удерживаеных токенов децентрализованных огранизаций.
\end{enumerate}

Коэфициент лейбла учитвает в подсчете:
\begin{enumerate}
	\item Совокупный доход с продажи лицензий ассоциированного актива;
	\item Доход с продажи лицензий за последнюю неделю;
	\item Искусственный коэфициент оценки лейбла SoundChain'ом (дополнительно депонирование средств, сжигание средств и т.д.)
\end{enumerate}
\begin{equation}
\end{equation}


%\subsection{Decentralized Marketplace}
%\label{tech-apps-marketplace}
%\end{multicols}
\end{document}
